\documentclass[a4paper, 10pt]{article}

\NeedsTeXFormat{LaTeX2e}

\makeindex

%---------------------------- PACKAGE INCLUSION -------------------------------
% This group renders characters clearer and more precise

\RequirePackage[bitstream-charter,cal,expert]{mathdesign}
\RequirePackage{latexsym}

\usepackage{geometry} % to change the page dimensions
\geometry{a4paper,
		  %showframe=true,
		  %margin=3em,
		  %a4paper,
		  %total={170mm,257mm},
		  top=4.15em,
		  left=3em,
		  right=3em,
		  bottom=3.39em
		  }
		  
\usepackage[default]{cantarell}
\usepackage{xspace}
%\usepackage[french]{babel}
\usepackage{paralist}
\usepackage[parfill]{parskip} % Activate to begin paragraphs with an empty line rather than an indent
\usepackage{listings} % for lstset definitions
\usepackage{graphicx} % support the \includegraphics command and options

\usepackage{verbatim}	

\usepackage{alltt}				% support for verbatim text with font specification. e.g: \textbf
\renewcommand{\ttdefault}{txtt}	   % support for verbatim text with font specification. e.g: \textbf

\usepackage{epsfig}
\usepackage{booktabs}

\usepackage{amsmath}
\newcommand{\mathbold}[1]{\text{\textbf{#1}}}


\usepackage{xcolor}
\definecolor{yerothColorOrange}{RGB}{242, 161, 0}   
\definecolor{yerothColorBlue}{RGB}{77 , 93 , 254}
\definecolor{yerothColorRed}{RGB}{254, 48 , 48}
\definecolor{yerothColorGray}{RGB}{198, 198, 198}
\definecolor{yerothColorDarkgray}{RGB}{60, 60 , 60}
\definecolor{yerothColorIndigo}{RGB}{83, 0, 125}
\definecolor{yerothColorGreen}{RGB}{2  , 160, 70}
\definecolor{forestgreen}{RGB}{2,160,70}    
\definecolor{mediumblue}{RGB}{7,43,205}    
\definecolor{firebrickred}{RGB}{178,34,34}
\definecolor{listingray}{gray}{0.9}
\definecolor{lbcolor}{rgb}{0.9,0.9,0.9}
\definecolor{darkgreen}{rgb}{0,0.35,0}
\definecolor{medgreen}{rgb}{0,0.5,0}
\definecolor{lightgreen}{rgb}{0.5,0.7,0.5}
\definecolor{pmcolour}{rgb}{0.5,0.7,0.5}
\definecolor{medgrey}{rgb}{0.6,0.6,0.6}
\definecolor{purplish}{rgb}{0.4,0,0.6}
\definecolor{brightred}{rgb}{1,0.2,0.2}

\usepackage{pagecolor}

\newcommand{\texlive}{\texttt{Texlive}\xspace}

\newcommand{\gdebi}{\texttt{gdebi}\xspace}

\newcommand{\expect}{\texttt{expect}\xspace}

\newcommand{\debian}{\texttt{Debian}\xspace}

\newcommand{\bash}{\texttt{bash}\xspace}

\newcommand{\qt}{\texttt{Qt}\xspace}

\newcommand{\mariadbserver}{\texttt{mariadb--server}\xspace}

\newcommand{\mariadbclient}{\texttt{mariadb--client}\xspace}

\newcommand{\yerothrd}{\textcolor{yerothColorGreen}
			{\textsc{\textcolor{yerothColorRed}{YEROTH}}$_{\text{r\&d}}$\xspace}}

\newcommand{\yerotherptroiszero}{\textcolor{yerothColorBlue}{\sc YEROTH--ERP--$3.0$}\xspace}

\newcommand{\yerotherpblack}{YEROTH--ERP--$3.0$\xspace}

\newcommand{\yerotherp}{\textcolor{yerothColorBlue}{\sc YEROTH--ERP--$3.0$}\xspace}

\newcommand{\erp}{ERP Software--System\xspace}

\newcommand{\myfullacademicname}{PR. XAVIER NOUMBISSI NOUNDOU\xspace}

\usepackage{hyperref}
\hypersetup{
    colorlinks,
	pagebackref,
    citecolor=medgreen,
    linkcolor=purplish,
    breaklinks,
    pdftex,
    bookmarks,
    plainpages=false,
	pdftitle={Installation Guide for \erp \yerotherp (standalone) by ''\myfullacademicname''},
    pdfauthor={PR. XAVIER NOUMBISSI NOUNDOU}
}

\usepackage{url}

\newcommand{\rootcommand}[1]{
\begin{center}
\textcolor{purplish}{#1\xspace}
\end{center}}

\newcommand{\emphbf}[1]{\emph{\textbf{#1}}\xspace}
\newcommand{\emphit}[1]{\emph{\textit{#1}}\xspace}
\newcommand{\mycheckmark}[1]{\textcolor{#1}{$\checkmark$}\xspace}
\newcommand{\mytimes}[1]{\textcolor{#1}{$\times$}\xspace}
\newcommand{\boldsc}[1]{\textbf{\textsc{#1}}\xspace}


\usepackage[T1]{fontenc}
\newcommand{\changefont}[3]{
\fontfamily{#1} \fontseries{#2} \fontshape{#3} \selectfont}
\changefont{cmss}{m}{n}

% Set font to avant-garde
%\renewcommand*\rmdefault{pag}

\pagenumbering{arabic}

%%%%%%%%%%%%%%%%%%%%SETTING HEADER AND FOOTER FOR 'REPORT CLASS'.%%%%%%%%%%%%%%%%%%%%%
\usepackage{lastpage}
\usepackage{fancyhdr}
\pagestyle{fancy}
\renewcommand{\headrulewidth}{0pt}
\fancyhf{}

\fancypagestyle{plain}{% copies "fancy" over "plain"
  \fancyfoot[C]{\thepage}% you can add edits that won't affect "fancy" but only "plain"
}

\fancypagestyle{OnlyFirstPage}{%
	\lhead{}
	\rhead{}
    \lfoot{}
}

\rhead{INSTALLATION GUIDE FOR \erp \yerotherpblack}
\lhead{\textbf{\yerothrd}}
\lfoot{}
\rfoot{}
\cfoot{\thepage\ DE \pageref{LastPage}}
%\cfoot{\thepage}


%%%%%%%%%%%%%%%%%%%%%%%%%%%%%%%%%%%%%%%%%%%%%%%%%%%%%%%%%%%%%%%%%%%%%%%%%%%


%Remove widows and orphants
\clubpenalty = 10000
\widowpenalty = 10000
\displaywidowpenalty = 10000

\begin{document}

\thispagestyle{OnlyFirstPage}

{\bf \Large \yerothrd} {| \sc \scriptsize \yerotherpblack installation guide}
\\ \line(1,0){540}

\vspace{2.0em}

\begin{center}
{\LARGE Installation Guide for \erp \yerotherpblack}
\end{center}

\vspace{2.0em}

\begin{center}
{\large \myfullacademicname}
\end{center}
%-------------------

\vspace{-0.5em}

% TABLE OF CONTENTS
\phantomsection
\addcontentsline{toc}{section}{\contentsname}
\begingroup
\tableofcontents
\endgroup

\vspace{0.25cm}

\section{Typography}

Within this guide, all commands written
in the following type setting:
	\begin{alltt}
		\rootcommand{command}
	\end{alltt}
are to be executed as ''super user'' (''root user'').

\section{Prerequisite Software}

\begin{table}[!htbp]
\centering
\begin{tabular}{l|r}
\textbf{Software}	&
\textbf{Versions}	\\ \hline
\debian				&
$10.0.0$ (buster)	\\ \hline
\gdebi				&
$0.9.5.7+nmu3$		\\ \hline
\expect				&
$5.45.4-2$			\\ \hline
\mariadbserver		&
$10.3$				\\ \hline
\mariadbclient		&
$10.3$				\\ \hline
\qt					&
$5.11.3$			\\ \hline	
\texlive			&
$2018.20190227-2$	\\
\end{tabular}
\caption{Prerequisite software for installation of \yerotherptroiszero.}
\label{tab:prerequisite-software}
\end{table}


\section{Files Required for Installation Procedure}

\begin{table}[!htbp]
\centering
\begin{tabular}{|l|} \hline
\textbf{Files}		\\ \hline
yeroth-erp-3.0-standalone.deb							\\ \hline
yeroth-erp-3-0-configure-mysql-server.sh				\\ \hline	
yeroth-erp-3-0-configure-mysql-server-set-root-pwd.exp	\\ \hline	
\end{tabular}
\caption{Files required for installation of \yerotherptroiszero.}
\label{tab:prerequisite-files}
\end{table}

Table~\ref{tab:prerequisite-files} illustrates files required
for installing \yerotherp.

\section{Installation Procedure}

Following steps have to be followed in order to have
a well functioning installation of \yerotherptroiszero:

\begin{enumerate} [1)]
	\item install OS \debian--buster
	\item install \gdebi and \expect software
	\item install database software \mariadbserver and \mariadbclient
	\item configure database software \mariadbserver
	\item install \yerotherptroiszero (\qt and \texlive are automatically installed).
\end{enumerate}

\subsection{Installation of \gdebi and \expect}

\begin{enumerate}[1)]
	\item Open a ''bash--terminal''
	\item type following command:
		\begin{alltt}
			\rootcommand{apt -y install gdebi expect}
		\end{alltt}
\end{enumerate} 

\subsection{Installation of \mariadbserver and \mariadbclient}

\begin{enumerate}[1)]
	\item Open a ''bash--terminal''
	\item type following command:
		\begin{alltt}
			\rootcommand{apt -y install mariadb-server mariadb-client}
		\end{alltt}		
\end{enumerate} 

\subsection{Configuration of \mariadbserver}

\begin{enumerate}[1)]
	\item Edit provided file ''yeroth-erp-3-0-configure-mysql-server-set-root-pwd.exp''
		and set \mariadbserver\ ''root'' password by replacing string 
		''admin1'' with a new password
	\item finally, run provided \bash file
		''yeroth-erp-3-0-configure-mysql-server.sh''	
\end{enumerate}

\subsection{Installation of \qt, \texlive, and of \yerotherptroiszero}

\begin{enumerate}[1)]
	\item Open a ''bash--terminal''

	\item then type following command:
		\begin{alltt}
			\rootcommand{gdebi -n yeroth-erp-3.0-standalone.deb}
		\end{alltt}

	This command automatically installs \qt, and \texlive.
		
	\item type in \mariadbserver\ ''root'' (administrative)
		password when prompted. This is required
		for setting up  database ''yeroth\_erp\_3''.
\end{enumerate}


\section{Standard Admistrative User Account 'admin'}


Afer \yerotherptroiszero successful installation,
the application is accessible using a standard
administrative user account with the following
credentials:

\begin{enumerate}[1)]
	\item user name: \textbf{admin}
	\item user password: \textbf{admin1}
\end{enumerate}

This administrative user account could be modified at your wish.

\end{document}
