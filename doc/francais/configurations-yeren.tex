\chapter{Les Configurations}\label{chap:congigurations-yeren}\index{configurations de \yeroth}
\index{les configurations \yeroth}

\chapintro{Ce chapitre d\'ecrit les diff\'erentes configurations
de \yeroth.}

\nxsection{Les configurations de \yeroth}

\yeroth a les configurations suivantes:
\begin{enumerate}[1)]
	\item la configuration \emphbf{standalone}
	\item la configuration \emphbf{acad\'emique}
	\item la configuration \emphbf{client}
	\item la configuration \emphbf{serveur}.\\
\end{enumerate}

\section{La configuration \emph{standalone} de \yeroth}
La configuration \emphbf{standalone} de \yeroth comprend
toutes les fonctionalit\'es. Elle est utilis\'ee pour
une boutique ou pour un d\'ep\^ot qui a besoin d'une seule
installation de \yeroth.

\section{La configuration \emph{client} de \yeroth}
La configuration \emph{client} fonctionne uniquement en
combinaison avec la configuration \emph{serveur} de \yeroth.

La configuration \emphbf{client} de \yeroth ne donne pas
acc\`es \`a la section \emphbf{Administration} du logiciel.
Aussi, elle ne permet pas l'acc\`es aux stocks des autres
localisations.

\section{La configuration \emph{serveur} de \yeroth}
La configuration \emphbf{serveur} fonctionne uniquement en
combinaison avec la configuration \emphbf{client} de \yeroth.

La configuration \emphbf{serveur} de \yeroth comprend
toutes les fonctionalit\'es et est utilis\'ee dans
le cadre d'une installation avec plusieurs \emphbf{clients}.

Lorsqu'une configuration dans la section \emphbf{param\`etres
de l'application} est modifi\'ee, le serveur envoi
un signal \`a chacun des \emphbf{clients} afin qu'ils
actualisent leurs param\`etres de configuration, en
les r\'ecup\'erant de la base de donn\'ees.
