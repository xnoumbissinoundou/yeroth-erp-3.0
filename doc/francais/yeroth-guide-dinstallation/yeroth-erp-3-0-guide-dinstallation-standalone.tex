\documentclass[a4paper, 10pt]{article}

\NeedsTeXFormat{LaTeX2e}

\makeindex

%---------------------------- PACKAGE INCLUSION -------------------------------
% This group renders characters clearer and more precise

\RequirePackage[bitstream-charter,cal,expert]{mathdesign}
\RequirePackage{latexsym}

\usepackage{geometry} % to change the page dimensions
\geometry{a4paper,
		  %showframe=true,
		  %margin=3em,
		  %a4paper,
		  %total={170mm,257mm},
		  top=4.15em,
		  left=3em,
		  right=3em,
		  bottom=3.39em
		  }
		  
\usepackage[default]{cantarell}
\usepackage{xspace}
\usepackage{paralist}
\usepackage[parfill]{parskip} % Activate to begin paragraphs with an empty line rather than an indent
\usepackage{listings} % for lstset definitions
\usepackage{graphicx} % support the \includegraphics command and options

\usepackage{verbatim}

\usepackage{alltt}				% support for verbatim text with font specification. e.g: \textbf
\renewcommand{\ttdefault}{txtt}	   % support for verbatim text with font specification. e.g: \textbf
		  
\usepackage{epsfig}
\usepackage{booktabs}

\newcommand{\texlive}{\texttt{Texlive}\xspace}

\newcommand{\gdebi}{\texttt{gdebi}\xspace}

\newcommand{\expect}{\texttt{expect}\xspace}

\newcommand{\debian}{\texttt{Debian}\xspace}

\newcommand{\bash}{\texttt{bash}\xspace}

\newcommand{\qt}{\texttt{Qt}\xspace}

\newcommand{\mariadbserver}{\texttt{mariadb--server}\xspace}

\newcommand{\mariadbclient}{\texttt{mariadb--client}\xspace}

\newcommand{\yerothrd}{\textsc{YEROTH~R\&D}\xspace}

\newcommand{\yerotherptroiszero}{\textcolor{yerenColorBlue}{\sc YEROTH--ERP--$3.0$}\xspace}

\newcommand{\yerotherp}{\textcolor{yerenColorBlue}{\sc YEROTH--ERP--$3.0$}\xspace}

\newcommand{\erp}{syst\`eme--logiciel ERP\xspace}

\newcommand{\myfullacademicname}{Dipl.--Inf. XAVIER NOUMBISSI NOUNDOU\xspace}

\usepackage{hyperref}
\hypersetup{
    colorlinks,
	pagebackref,
    citecolor=medgreen,
    linkcolor=purplish,
    breaklinks,
    pdftex,
    bookmarks,
    plainpages=false,
	pdftitle={Guide d'installation pour le \erp \yerotherp (standalone) par: ''\myfullacademicname''},
    pdfauthor={XAVIER NOUMBISSI NOUNDOU}
}

\usepackage{url}

\usepackage{xcolor}
\definecolor{yerenColorOrange}{RGB}{242, 161, 0}   
\definecolor{yerenColorBlue}{RGB}{77 , 93 , 254}
\definecolor{yerenColorRed}{RGB}{254, 48 , 48}
\definecolor{yerenColorGray}{RGB}{198, 198, 198}
\definecolor{yerenColorDarkgray}{RGB}{60, 60 , 60}
\definecolor{yerenColorIndigo}{RGB}{83, 0, 125}
\definecolor{yerenColorGreen}{RGB}{2  , 160, 70}
\definecolor{forestgreen}{RGB}{2,160,70}    
\definecolor{mediumblue}{RGB}{7,43,205}    
\definecolor{firebrickred}{RGB}{178,34,34}
\definecolor{listingray}{gray}{0.9}
\definecolor{lbcolor}{rgb}{0.9,0.9,0.9}
\definecolor{darkgreen}{rgb}{0,0.35,0}
\definecolor{medgreen}{rgb}{0,0.5,0}
\definecolor{lightgreen}{rgb}{0.5,0.7,0.5}
\definecolor{pmcolour}{rgb}{0.5,0.7,0.5}
\definecolor{medgrey}{rgb}{0.6,0.6,0.6}
\definecolor{purplish}{rgb}{0.4,0,0.6}
\definecolor{brightred}{rgb}{1,0.2,0.2}

\usepackage{pagecolor}

\newcommand{\rootcommand}[1]{
\begin{center}
\textcolor{purplish}{#1\xspace}
\end{center}}

\newcommand{\yeren}{\textsc{yeroth-erp-3.0}\xspace}

\newcommand{\emphbf}[1]{\emph{\textbf{#1}}\xspace}
\newcommand{\emphit}[1]{\emph{\textit{#1}}\xspace}
\newcommand{\mycheckmark}[1]{\textcolor{#1}{$\checkmark$}\xspace}
\newcommand{\mytimes}[1]{\textcolor{#1}{$\times$}\xspace}
\newcommand{\boldsc}[1]{\textbf{\textsc{#1}}\xspace}

\usepackage[T1]{fontenc}
\newcommand{\changefont}[3]{
\fontfamily{#1} \fontseries{#2} \fontshape{#3} \selectfont}
\changefont{cmss}{m}{n}

% Set font to avant-garde
%\renewcommand*\rmdefault{pag}

\usepackage[french]{babel}
\usepackage{fancyhdr}
\pagestyle{fancy}
\renewcommand{\headrulewidth}{0pt}
\lhead{}
\rhead{}
\rfoot{{\small version du --~\today~--}}
\cfoot{\thepage}

%Remove widows and orphants
\clubpenalty = 10000
\widowpenalty = 10000
\displaywidowpenalty = 10000

\begin{document}

\pagenumbering{arabic}

\title{
\vspace{-1.65em}
\textcolor{medgreen}{\textsc{Guide d'installation \\
										pour le \\
									 \erp \\ \vspace{1em}
									 \yerotherp \\ 
									 \hspace{0.6em} \textcolor{yerenColorBlue}{(standalone)} }}
									 \author{\myfullacademicname}
}

\date{} 
\maketitle
\thispagestyle{fancy}
%\bigskip 
%-------------------

\vspace{-0.5em}

% TABLE OF CONTENTS
\phantomsection
\addcontentsline{toc}{section}{\contentsname}
\begingroup
\tableofcontents
\endgroup


\vspace{0.25cm}

\section{Typographie}

Toute les commandes dans ce guide d\'ecrite
de la mani\`ere suivante:
	\begin{alltt}
		\rootcommand{commande}
	\end{alltt}
sont \`a \^etre ex\'ecuter en tant
''super utilisateur'' (''utilisateur root'').

\section{Logiciels Pr\'erequis}

\begin{table}[!htbp]
\centering
\begin{tabular}{l|r}
\textbf{Logiciels}	&
\textbf{Versions}	\\ \hline
\debian				&
$10.0.0$ (buster)	\\ \hline
\gdebi				&
$0.9.5.7+nmu3$		\\ \hline
\expect				&
$5.45.4-2$			\\ \hline
\mariadbserver		&
$10.3$				\\ \hline
\mariadbclient		&
$10.3$				\\ \hline
\qt					&
$5.11.3$			\\ \hline	
\texlive			&
$2018.20190227-2$	\\ 			
\end{tabular}
\caption{Logiciels requis pour l'installation de \yerotherptroiszero.}
\label{tab:dependance-logiciel}
\end{table}

\section{Fichiers Requis pour la Proc\'edure D'installation}

\begin{table}[!htbp]
\centering
\begin{tabular}{|l|} \hline
\textbf{Fichiers}		\\ \hline
yeroth-erp-3.0-standalone.deb							\\ \hline
yeroth-erp-3-0-configure-mysql-server.sh				\\ \hline	
yeroth-erp-3-0-configure-mysql-server-set-root-pwd.exp	\\ \hline	
\end{tabular}
\caption{Fichiers requis pour l'installation de \yerotherptroiszero.}
\label{tab:required-files}
\end{table}

Le tableau~\ref{tab:required-files} illustre les fichiers
requis pour l'installation de \yerotherp.

\section{Proc\'edure D'installation}

Il faut suivre les \'etapes suivantes pour obtenir
une installation de \yerotherptroiszero qui fonctionne
sans probl\`emes:

\begin{enumerate} [1)]
	\item installer le syst\`eme d'exploitation \debian--buster
	\item installer les logiciels \gdebi et \expect
	\item installer les logiciels \mariadbserver et \mariadbclient
	\item configurer le logiciel \mariadbserver
	\item installer enfin \yerotherptroiszero (\qt et \texlive sont install\'e automatiquement).
\end{enumerate}

\subsection{Installation de \gdebi et \expect}

\begin{enumerate}[1)]
	\item Ouvrir un terminal ''\bash''
	\item Taper la commande suivante:
		\begin{alltt}
			\rootcommand{apt -y install gdebi expect}
		\end{alltt}
\end{enumerate} 

\subsection{Installation de \mariadbserver et \mariadbclient}

\begin{enumerate}[1)]
	\item Ouvrir un terminal ''\bash''
	\item Taper la commande suivante:
		\begin{alltt}
			\rootcommand{apt -y install mariadb-server mariadb-client}
		\end{alltt}		
\end{enumerate} 

\subsection{Configuration de \mariadbserver}

\begin{enumerate}[1)]
	\item Editer le fichier (donn\'e avec \yerotherptroiszero)
		''yeroth-erp-3-0-configure-mysql-server-set-root-pwd.exp''
		et entrer y le mot de passe souhait\'e pour
		l'utilisateur ''root'' (l'administrateur) du logiciel \mariadbserver
		\`a la place du mot de passe ''admin1''
		
	\item enfin ex\'ecuter le fichier \bash (donn\'e avec \yerotherptroiszero)
		''yeroth-erp-3-0-configure-mysql-server.sh''.\\	
\end{enumerate}

\subsection{Installation de \qt, \texlive, et de \yerotherptroiszero}

\begin{enumerate}[1)]
	\item Ouvrir un terminal ''\bash''
	\item ensuite entrer y la commande suivante:
		\begin{alltt}
			\rootcommand{gdebi -n yeroth-erp-3.0-standalone.deb}
		\end{alltt}
		
		Cette commande installe automatique \qt et \texlive.
		
	\item Entrer le mot de passe de l'utilisateur ''root''
		du logiciel \mariadbserver lorsque cela vous
		sera demand\'e
		
		Ceci est requis pour l'installation de la base
		de donn\'ees ''yeroth\_erp\_3''.\\
\end{enumerate}


\section{L'utilisateur Standard (Administrateur) 'admin'}

Apr\`es l'installation avec succ\`es de \yerotherptroiszero,
l'application est accessible par l'utilisation
d'un compte utilisateur standard avec le r\^ole
d'administrateur en utilisant les donn\'ees suivantes:
\begin{enumerate}[1)]
	\item nom d'utilisateur: \textbf{admin}
	\item mot de passe: \textbf{admin1}\\
\end{enumerate}

Ce compte \emph{administrateur} peut \^etre modifi\'e
selon votre volont\'e.

\end{document}
