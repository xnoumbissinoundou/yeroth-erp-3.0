\chapter{Introduction}\label{chap:introduction}

\yeren est un syst\`eme logiciel de gestion des
stocks et de gestion des ventes. Il permet
d'ex\'ecuter des mouvements de stocks, et de
vendre des articles en stocks.

Une entreprise doit poss\'eder au minimum d'un stock
d'articles, d'un d\'ep\^ot ou d'une boutique pour utiliser
\yeren de fa\c{c}on efficace.\\

\begin{figure}[!htpb]
	\centering
	\includegraphics[scale=0.63]{images/architecture-enterprise-yeren.png}
	\caption{Un mod\`ele d'architecture d'une entreprise}\label{fig:architecture-enterprise-yeren}
\end{figure}

La figure~\ref{fig:architecture-enterprise-yeren} illustre
un mod\`ele g\'en\'erique d'entreprise o\`u les d\'ep\^ots
et les boutiques de l'entreprise sont en communication.

Les activit\'es principales de l'entreprise sont les suivantes:
\begin{enumerate}[1)]
	\item \emphbf{les sorties de stocks:} sortie d'articles
		d'une unit\'e (boutique ou d\'ep\^ot) pour r\'eception par un client
	\item \emphbf{les transferts de stocks:} mouvement d'articles
		d'une unit\'e vers une autre unit\'e
	\item \emphbf{les ventes d'articles:} un client ach\`ete
		des articles qui lui sont ensuite remis.\\
\end{enumerate}

\yeren permet d'accomplir les t\^aches de gestion
des stocks et de gestion des ventes suivantes:
\begin{enumerate}[1)]
	\item \myenumitem{cr\'eer des alertes sur des p\'eriodes de temps}	
	\item \myenumitem{cr\'eer des alertes sur des quantit\'e en stocks}	
	\item \myenumitem{entrer un stock}
	\item \myenumitem{lister les stocks}
	\item \myenumitem{modifier un stock}
	\item \myenumitem{supprimer un stock}	
	\item \myenumitem{rechercher des articles ou des stocks}
	\item \myenumitem{transf\'erer des articles ou des stocks}	
	\item \myenumitem{vendre des articles}	
	\item \myenumitem{visualiser les \'etats de transactions d'articles
		(sorties ou transferts de stocks)}
	\item \myenumitem{visualiser les \'etats de ventes d'articles}.\\
\end{enumerate}

\newpage

\section{Acc\`es au manuel de l'utilisateur}

La figure~\ref{fig:fenetre-principale-utilisateur-non-enregistre}
illustre la fen\^etre d'accueil de \yeren sans aucun utilisateur
enregistr\'e.\\

\begin{figure}[!htbp]
\centering
\includegraphics[scale=0.63]{images/yeren-fenetre-principale.png}
\caption{La fen\^etre d'acceuil sans aucun utilisateur enregistr\'e.}
\label{fig:fenetre-principale-utilisateur-non-enregistre}
\end{figure}

Il est requis qu'un utilisateur soit enregistr\'e
dans \yeren afin d'avoir acc\`es au manuel de l'utilisateur.

L'utilisateur de \yeren doit accomplir les op\'erations
suivantes afin d'avoir acc\`es au manuel de l'utilisateur:
\begin{enumerate}[1)]
	\item \`a partir de la fen\^etre d'accueil
		(voir figure~\ref{fig:fenetre-principale-utilisateur-non-enregistre}),
		cliquez sur le menu d\'eroulant '\textbf{Aide}'
	\item ensuite cliquez sur le lien '\textbf{Manuel de l'utilisateur (PDF)}'.
\end{enumerate}

\section{Structure de ce manuel de l'utilisateur}
Ce manuel de  l'utilisateur de \yeren est structur\'e
comme suit:

\begin{itemize}[\mycheckmark{purplish}]
	\item le chapitre~\ref{chap:utilisateurs} d\'ecrit
	les utilisateurs de \yeren et leurs \roles. 
	     
	\item le chapitre~\ref{chap:gestion-stocks} explicite
	les fonctionalit\'es de gestion des stocks
	
	\item le chapitre~\ref{chap:systeme-dalertes}
	pr\'esente le syst\`eme d'alertes sur les stocks
	
	\item le chapitre~\ref{chap:vendre} d\'ecrit comment
	conclure des ventes d'articles
	
	\item le chapitre~\ref{chap:sortir-articles} d\'ecrit
	comment proc\'eder \`a des sorties et transferts de stocks
	
	\item le chapitre~\ref{chap:vente} explicite comment
	rechercher et imprimer les \'etats de ventes d'articles
	
	\item le chapitre~\ref{chap:etats-des-sorties} explicite
	comment rechercher et imprimer les \'etats de sorties ou
	transferts d'articles
	
	\item le chapitre~\ref{chap:tableaux-de-bord} discute
	de la recherche et de la g\'en\'eration des rapports
	commerciaux de l'entreprise
	
	\item le chapitre~\ref{chap:informations-generales}
	explique comment avoir acc\`es aux d\'etails de
	l'utilisateur enregistr\'e, aux informations commerciales
	de l'entreprise, et enfin \`a la version de \yeren que
	l'on utilise
	
	\item le chapitre~\ref{chap:administration-logiciel}
	traite de l'administration du logiciel

	\item le chapitre~\ref{chap:problemes-connues}
	discute des probl\`emes connues du syst\`eme--logiciel
	
	\item enfin, le chapitre~\ref{chap:conclusion} conclut
	ce manuel d'utilisation.
\end{itemize}

