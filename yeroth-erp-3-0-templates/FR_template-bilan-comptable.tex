\documentclass[10pt,YEROTHPAPERSPEC,landscape]{article} % use larger type; default would be 10pt
\NeedsTeXFormat{LaTeX2e}
\makeindex

%---------------------------- PACKAGE INCLUSION -------------------------------
% This group renders characters clearer and more precise

\RequirePackage[bitstream-charter,cal,expert]{mathdesign}
\RequirePackage{charter}
\RequirePackage{helvet}
\RequirePackage{makeidx}
\RequirePackage{latexsym}

\usepackage{geometry}
\geometry{YEROTHPAPERSPEC,
		  top=3.5em,
		  left=3em,
		  right=3em,
		  bottom=3.39em
		  }
		  
\usepackage{graphicx}
\usepackage{adjustbox}
\usepackage{xspace}
\usepackage[parfill]{parskip} % Activate to begin paragraphs with an empty line rather than an indent
\usepackage{paralist} % very flexible & customisable lists (eg. enumerate/itemize, etc.)
\usepackage{listings} % for lstset definitions
\usepackage{url}
\usepackage{subfig} % make it possible to include more than one captioned figure/table in a single float
\usepackage{epsfig}
\usepackage{gensymb}
\usepackage{textcomp}
\usepackage{booktabs}

\usepackage{amsmath}
\newcommand{\mathbold}[1]{\text{\textbf{#1}}}

\usepackage{xcolor}
\definecolor{yerothColorOrange}{RGB}{242, 161, 0}   
\definecolor{yerothColorBlue}{RGB}{77 , 93 , 254}
\definecolor{yerothColorRed}{RGB}{254, 48 , 48}
\definecolor{yerothColorGray}{RGB}{198, 198, 198}
\definecolor{yerothColorDarkgray}{RGB}{60, 60 , 60}
\definecolor{yerothColorIndigo}{RGB}{83, 0, 125}
\definecolor{yerothColorGreen}{RGB}{2  , 160, 70}
\definecolor{forestgreen}{RGB}{2,160,70}    
\definecolor{mediumblue}{RGB}{7,43,205}    
\definecolor{firebrickred}{RGB}{178,34,34}
\definecolor{listingray}{gray}{0.9}
\definecolor{lbcolor}{rgb}{0.9,0.9,0.9}
\definecolor{darkgreen}{rgb}{0,0.35,0}
\definecolor{medgreen}{rgb}{0,0.5,0}
\definecolor{lightgreen}{rgb}{0.5,0.7,0.5}
\definecolor{pmcolour}{rgb}{0.5,0.7,0.5}
\definecolor{medgrey}{rgb}{0.6,0.6,0.6}
\definecolor{purplish}{rgb}{0.4,0,0.6}
\definecolor{brightred}{rgb}{1,0.2,0.2}

\usepackage{hyperref}
\hypersetup{
    colorlinks,
    pagebackref,
    citecolor=medgreen,
    linkcolor=purplish,
    breaklinks,
    pdftex,
    bookmarks,
    plainpages=false,
    pdftitle={Bilan comptable du YEROTHBILANCOMPTABLEDEBUT
       au YEROTHBILANCOMPTABLEFIN imprim\'e par YEROTHNOMUTILISATEUR (YEROTHENTREPRISE)},
    pdfauthor={YEROTHENTREPRISE (YEROTHNOMUTILISATEUR)}
}

% format two pieces of text, one left aligned and one right aligned
\newcommand{\headerrow}[2]
{\begin{tabular*}{\linewidth}{l@{\extracolsep{\fill}}r}
	#1 &
	#2 \\
\end{tabular*}}

\newcommand{\emphbold}[1]{\textbf{\emph{#1}}\xspace}

\pagenumbering{gobble}

\begin{document}
\bigskip

\headerrow
	{\emphbold{YEROTHENTREPRISE}}
	{\emph{\textbf{Num\'ero de contribuable:} YEROTHCONTRIBUABLENR}}
\headerrow
	{\emphbold{YEROTHACTIVITESENTREPRISE}}
	{\emph{\textbf{Num\'ero de compte bancaire:} YEROTHCOMPTEBANCAIRENR,}}
\headerrow
	{\emphbold{\'Email: YEROTHEMAIL}}
	{\emph{YEROTHAGENCECOMPTEBANCAIRE}}
\headerrow
	{\emphbold{T\'el.: YEROTHTELEPHONE}}
	{}
\headerrow
	{\emphbold{B.P.: YEROTHBOITEPOSTALE, YEROTHVILLE}}
	{}
	
\hrule

\headerrow
	{}
	{\textbf{YEROTHDATE}}

\section*{Bilan comptable du YEROTHBILANCOMPTABLEDEBUT au YEROTHBILANCOMPTABLEFIN}

\textbf{Imprim\'e par:} YEROTHNOMUTILISATEUR\\
\textbf{Heure d'impression:} YEROTHHEUREDIMPRESSION\\

\vspace{0.3cm}

\vspace{1cm}
\textbf{ENTR\'EES}
\begin{table}[!htbp]
\begin{tabular}{lrr}
ventes  						&  YEROTHBILANCOMPTABLEVENTESDEVISE  			& [V] \\ 
versements aux comptes clients  &  YEROTHBILANCOMPTABLEVERSEMENTSCLIENTSDEVISE  & \\ \hline
\textbf{Total entr\'ees}  		&  YEROTHBILANCOMPTABLETOTALENTREESDEVISE & [TE] 		\\ 
\end{tabular}
\end{table}


\vspace{0.5cm}
\textbf{SORTIES}
\begin{table}[!htbp]
\begin{tabular}{lrr}
achats d'articles  						& YEROTHBILANCOMPTABLEACHATSDEVISE  				& \\
payements pour ''FID\'ELIT\'E CLIENTS'' & YEROTHBILANCOMPTABLEPROGRAMMEFIDELITECLIENTS  	& \\  
charges (d\'epenses financi\`eres) 		& YEROTHBILANCOMPTABLECHARGESDEPENSESFINANCIERES  	& \\ 
TVA engrang\'ee  						& YEROTHBILANCOMPTABLETVAENGRANGE 					&  \\
dette aux comptes clients\footnotemark  & YEROTHBILANCOMPTABLEDETTECLIENTELLEDEVISE 		&  \\ \hline
\textbf{Total sorties}  		&  YEROTHBILANCOMPTABLETOTALSORTIESDEVISE & [--TS] 			\\ 
\end{tabular}
\end{table}


\vspace{1cm}

\begin{table}[!htbp]
\centering
\begin{tabular}{lrr}
\textbf{B\'en\'efice sur les ventes}  &  YEROTHBILANCOMPTABLEBENEFICEDEVISE &  \\ \hline
\textbf{Balance}  			&  YEROTHBILANCOMPTABLEBALANCEDEVISE  & [TE -- TS]\\ \hline
\textbf{Chiffre d'affaire}  &  YEROTHBILANCOMPTABLECHIFFREDAFFAIREDEVISE  & [TE]\\ 
\end{tabular}
\end{table}

\footnotetext{Tout temps compris.}
\end{document}